\setcounter{page}{1}    % 将页码计数器设置为 1

% ===============================================================
%
% 问题重述与分析
%
% ===============================================================

\section{问题重述与分析}

\subsection{问题背景}

互联网迅猛发展,线上教育平台这种新型教育模式逐渐兴起。各种基于互联网的教育模式渐渐的发展起来了。利用互联网的高度便利性和自定义性,因材施教的程度得到了进一步发展。为了进一步实现用户的个性化学习,某\linebreak MOOC在线教育平台提供了个性化题库的功能。该题库系统会记录用户的学习过程,而自动生成对应的课后习题。但该系统目前来说还存在着很大缺陷。

题目系统为了实现个性化试题,主要是实现两个子功能:\textbf{相似度评估系统}和\textbf{难度评估系统}。

目前而言,这两个系统都有明显的缺陷。

\subsubsection{相似度评估系统}

该系统中,评判两个题目之间相似度主要依据是\textbf{题干文字}和\textbf{事先标注题目的知识点信息}。前者无法应对不同表述但是解法相同的题目,后者与知识点划分方式相关,难以达到真正的拓展练习的地步,这急需要改进。

\subsubsection{难度评估系统}

该系统中,判断题目难度的依据主要是\textbf{考试的类型}和\textbf{教师的主观经验}。这两种方式的局限性都太大了,前者只能够判断考试试题的难度,然而还有更多的题目是不会出现在考试试题中的;后者太主观了,不同的老师可能会给出完全不同的两种回答。并且,一个题目难度和题面的表达、学习者的状态、学习者的知识储备等等因素有关。所以,该系统仍然需要进一步改进。

\subsection{问题分析}
 
\subsubsection{关于相似度和难度评估的研究现状}

\subsubsection{基于NLP理论和DBSCAN聚类的相似度检验}

\subsubsection{}

% ===============================================================
%
% 模型假设与符号说明
%
% ===============================================================

\section{模型假设与符号说明}