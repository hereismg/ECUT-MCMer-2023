\setcounter{page}{1}    % 将页码计数器设置为 1

% ===============================================================
%
% 问题重述与分析
%
% ===============================================================

\section{问题重述与分析}

\subsection{问题背景}

互联网迅猛发展,线上教育平台这种新型教育模式逐渐兴起。各种基于互联网的教育模式渐渐的发展起来了。利用互联网的高度便利性和自定义性,因材施教的程度得到了进一步发展。为了进一步实现用户的个性化学习,某\linebreak MOOC在线教育平台提供了个性化题库的功能。该题库系统会记录用户的学习过程,而自动生成对应的课后习题。但这一习题目前来说还存在着很大的缺陷。

题目系统为了实现个性化试题,主要是实现两个子功能:\textbf{相似度评估系统}和\textbf{难度评估系统}。

目前而言,这两个系统都有明显的缺陷。

\subsubsection{相似度评估系统}

该系统中,评判两个习题之间相似度主要依据是\textbf{题干文字}和\textbf{事先标注题目的知识点信息}。

\subsubsection{难度评估系统}



% ===============================================================
%
% 模型假设与符号说明
%
% ===============================================================

\section{模型假设与符号说明}