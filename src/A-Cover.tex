\thispagestyle{empty}   % 定义起始页的页眉页脚格式为 empty —— 空,也就没有页眉页脚

\begin{center}
    \textbf{\fontsize{30}{1.5} 2023年东华理工大学数学建模竞赛}
\end{center}

我们完全明白,在竞赛开始后参赛队员不能以任何方式(包括电话、电子邮件、网上咨询等)与本队以外的任何人(包括指导教师)研究、讨论与赛题有关的问题。

我们知道,抄袭别人的成果是违反竞赛规则的, 如果引用别人的成果或其它公开的资料(包括网上查到的资料),必须按照规定的参考文献的表述方式在正文引用处和参考文献中明确列出。

我们郑重承诺,严格遵守竞赛规则,以保证竞赛的公正、公平性。如有违反竞赛规则的行为,我们愿意承担由此引起的一切后果。

参赛题号(从A/B/C中选择一项填写):

参赛队报名号:53

是否愿意参加暑假数学建模集训:愿意(愿意或不愿意)\newline


{\fontsize{20}{2}参赛队员信息:}

\begin{table}[h]
    \centering
    \begin{tabular}{|c|c|c|c|}
    \hline
       & 参赛队员1      & 参赛队员2      & 参赛队员3      \\ \hline
    姓名 & 黄镁豪        & 董政         & 谢宗晟        \\ \hline
    学号 & 2021213357 & 2021213195 & 2022213124 \\ \hline
    学院 & 软件学院       & 软件学院       & 软件学院       \\ \hline
    专业 & 软件工程       & 软件工程       & 软件工程       \\ \hline
    \end{tabular}
\end{table}

% ==============================================
% 接下来时论文正文
% ==============================================

\newpage

\begin{center}
    \textbf{\fontsize{20}{1.5}小学数学应用题相似性度量及难度评估}

    \textbf{摘要}
\end{center}
