% \thispagestyle{empty}   % 定义起始页的页眉页脚格式为 empty —— 空,也就没有页眉页脚

% \begin{center}
%     \textbf{\fontsize{30}{1.5} 2023年东华理工大学数学建模竞赛}
% \end{center}

% 我们完全明白,在竞赛开始后参赛队员不能以任何方式(包括电话、电子邮件、网上咨询等)与本队以外的任何人(包括指导教师)研究、讨论与赛题有关的问题。

% 我们知道,抄袭别人的成果是违反竞赛规则的, 如果引用别人的成果或其它公开的资料(包括网上查到的资料),必须按照规定的参考文献的表述方式在正文引用处和参考文献中明确列出。

% 我们郑重承诺,严格遵守竞赛规则,以保证竞赛的公正、公平性。如有违反竞赛规则的行为,我们愿意承担由此引起的一切后果。

% 参赛题号(从A/B/C中选择一项填写):

% 参赛队报名号:53

% 是否愿意参加暑假数学建模集训:愿意(愿意或不愿意)\newline


% {\fontsize{20}{2}参赛队员信息:}

% \begin{table}[h]
%     \centering
%     \begin{tabular}{|c|c|c|c|}
%     \hline
%        & 参赛队员1      & 参赛队员2      & 参赛队员3      \\ \hline
%     姓名 & 黄镁豪        & 董政         & 谢宗晟        \\ \hline
%     学号 & 2021213357 & 2021213195 & 2022213124 \\ \hline
%     学院 & 软件学院       & 软件学院       & 软件学院       \\ \hline
%     专业 & 软件工程       & 软件工程       & 软件工程       \\ \hline
%     \end{tabular}
% \end{table}

% ==============================================
%
% 论文标题、摘要、目录
%
% ==============================================

\newpage
\thispagestyle{empty}

\begin{center}
    \textbf{\fontsize{20}{1.5}小学数学应用题相似性度量及难度评估}

    \textbf{摘要}
\end{center}

为了提高学习者的学习效率,对题目相似性检验和难度评估的研究就显得十分重要。然而当前的相似性检验方法和难度评估手段有很多不足之处。为了弥补当前评判方法的不足,本文采用\textbf{词袋模型}和\textbf{LDA模型}判断题目相似性,用\textbf{模糊数学}的方法评判题目难度,定义难度系数。

\textbf{对于任务一——题目相似度检验,本文建立词袋聚类模型判断题目相似度。}在改模型中,首先通过\textbf{词袋模型}建立“词典”,得到题目的文本向量。提取关键字以后,通过\textbf{LDA模型}给出最后的结果。对于LDA模型而言,需要选择合理的参数,才会有较好的拟合效果。最终,通过上面的处理之后,就能给原本的题目词组分类,已达到相似检验的目的。

\textbf{对于任务二——难度评估,本文通过基于模糊数学的模糊综合评价定义题目的难度系数。}为了只考虑题目自身的难度而不考虑外在于因素对题目难度的影响,在模糊综合评价的过程中,将考虑“知识点数量”、“数学运算的难易”、“试题背景的复杂性”和“个人主观评分”。其中,四个评价维度的重要程度依次递减。

\textbf{对于任务三——给附件1中的100道题目分类,本文通过一系列计算得到计算结果,见"附件-题目分类.xlsx"。}任何两个题目之间的关系都不是非黑即白的二元对立,而是相似程度的差别。通过对相似度的计算,依次为基础给这些题目分类。既可以得到最终的结果。实际上,该模型适用于更大规模的题库。更多的题库可以让模型更加适配新的题目。

\textbf{对于任务四——分析附件2中10个题目的难度,本文通过一系列计算后得到计算结果,见“附件-题目难度系数计算.zip”。}该任务是对于前文模型的实际应用,在该任务中,首先由若干位专家从四个维度给题目评估,然后汇总数据,进行相关的计算后,得出每道题目的难度系数。
\newline
\newline
\textbf{关键词}:余弦相似度 \quad LDA模型 \quad 题目相似度评估 \quad 题目难度评估 \quad 模糊数学

% 下面添加一个目录,同样定义目录页眉页脚格式为 empty —— 空
\newpage
\thispagestyle{empty}
\tableofcontents